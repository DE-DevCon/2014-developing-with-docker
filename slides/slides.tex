\documentclass{beamer}
\usepackage[latin1]{inputenc}
\usepackage{listings}
\usepackage{multicol}
\usetheme{Berlin}
\setbeamercolor{palette secondary}{use=seahorse}
\usecolortheme{seahorse}
\title[Developing with Docker]{Developing with Docker}
\author{\hspace{12pt}Robert Bittle\hspace{12pt}\\\hspace{12pt}robert.bittle@dominionenterprises.com\hspace{12pt}\\\hspace{12pt}github.com/guywithnose\hspace{12pt}}
\date{December 12, 2014}
\setbeamertemplate{itemize items}[circle]
\begin{document}

    \begin{frame}
    \titlepage
    \end{frame}

    % \section{Intro}
    % \begin{frame}{Who am I?}
    %     \begin{itemize}
    %         % Be quick, maybe cut this slide
    %         \item robert.bittle@dominionenterprises.com
    %         \item github.com/guywithnose
    %     \end{itemize}
    % \end{frame}
    \section{Docker}
    \subsection{What is Docker?}
    \begin{frame}{Similar to virtual machines...}
        % Similar to virtual machines, but without the overhead
        % Keep this high level use an image
        \begin{figure}[htpb]
            \centering
            \includegraphics[width=0.8\linewidth]{VM.jpg}
        \end{figure}
    \end{frame}
    \begin{frame}{...only simpler}
        % Similar to virtual machines, but without the overhead
        % Keep this high level use an image
        \begin{figure}[htpb]
            \centering
            \includegraphics[width=0.8\linewidth]{Docker.jpg}
        \end{figure}
    \end{frame}
    % Basic comands
    \subsection{Working with images and containers}
    \begin{frame}{How to run a container}
        \alert<+>{docker run\alt<+->{$\backslash$\\}{ nginx}}
        \alert<.>{\only<.->{\textendash\textendash detach $\backslash$\\}}
        \alert<+>{\only<.->{\textendash\textendash volume \$(pwd):/usr/local/nginx/html $\backslash$\\}}
        \alert<+>{\only<.->{\textendash\textendash publish 80:80 $\backslash$\\}}
        \only<2->{nginx}
    \end{frame}
    \begin{frame}{The Dockerfile}
        % The emphasis here is creating a reproducible environment
        \alert<+>{}
        \alert<+>{FROM ubuntu:14.10\\}
        \alert<+>{RUN apt-get update \&\& apt-get install -y php5-fpm\\}
        \alert<+>{CMD /usr/sbin/php5-fpm --nodaemonize\\}
    \end{frame}
    \begin{frame}{Building the Dockerfile}
        docker build -t php5-fpm .
    \end{frame}
    \begin{frame}{How to link images}
        % Ideally a docker container should run only one process This is not
        % strictly necessary, but it is the way Docker was designed to work If
        % your system requires more than one process (nginx, php-fpm, mongo,
        % etc...), you will need to link your containers together so they can
        % communicate.
    \end{frame}
    \section{Fig}
    \subsection{What is Fig?}
    \begin{frame}{}
        % Useful for running containers in groups
        % This is essential for development environments
        % Keep this high level use the same image
        % Talk about this being incorporated into docker
    \end{frame}
    \subsection{}
    \begin{frame}{How to build a group}
    \end{frame}
    \begin{frame}{Run, stop, rebuild}
    \end{frame}
    % Give a full demo of an nginx, php-fpm, mongo system
    \section{Docker in production}
    \subsection{}
    \begin{frame}{Modifications}
    % ADD code instead of mounting as a volume
    \end{frame}
\end{document}
